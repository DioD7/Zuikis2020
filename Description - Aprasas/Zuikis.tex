%% LyX 2.3.5.2 created this file.  For more info, see http://www.lyx.org/.
%% Do not edit unless you really know what you are doing.
\documentclass[12pt]{article}
\usepackage{amsmath}
\usepackage{amsthm}
\usepackage{tgtermes}
\usepackage{newtxmath}
\renewcommand{\familydefault}{\rmdefault}
\usepackage[T1,L7x]{fontenc}
\usepackage[utf8]{inputenc}
\usepackage[a4paper]{geometry}
\geometry{verbose,tmargin=2cm,bmargin=2cm,lmargin=3cm,rmargin=1.5cm}
\usepackage{fancyhdr}
\pagestyle{fancy}
\usepackage{color}
\usepackage[lithuanian]{babel}
\usepackage{enumitem}
\usepackage{cancel}
\usepackage{setspace}
\onehalfspacing
\usepackage[unicode=true,pdfusetitle,
 bookmarks=true,bookmarksnumbered=true,bookmarksopen=false,
 breaklinks=false,pdfborder={0 0 1},backref=false,colorlinks=false]
 {hyperref}
\hypersetup{
 pdfstartview=FitH, bookmarksopen=true, hidelinks=true,  pdfdisplaydoctitle=true, pdfpagemode=UseNone}

\makeatletter

%%%%%%%%%%%%%%%%%%%%%%%%%%%%%% LyX specific LaTeX commands.
%% Because html converters don't know tabularnewline
\providecommand{\tabularnewline}{\\}

%%%%%%%%%%%%%%%%%%%%%%%%%%%%%% Textclass specific LaTeX commands.
\newlength{\lyxlabelwidth}      % auxiliary length 

\@ifundefined{date}{}{\date{}}
%%%%%%%%%%%%%%%%%%%%%%%%%%%%%% User specified LaTeX commands.
% Dokumento autoriai Jevgenij Chmeliov ir Andrius Gelžinis, (c) 2018
\usepackage{tocloft}
\usepackage{indentfirst}
\frenchspacing
\usepackage{cite}
\usepackage{upgreek}
\usepackage{siunitx} % dydžių vienetams
\sisetup{
redefine-symbols = true,
output-decimal-marker = {,},
exponent-product = \cdot,
inter-unit-product = \ensuremath{{}\cdot{}}}


\usepackage[labelsep=space]{caption}
\DeclareCaptionLabelFormat{pav}{#2~#1}
\captionsetup[figure]{labelformat=pav, name=pav.}
\captionsetup[table]{labelformat=pav, name=lentelė.}

\DeclareMathOperator{\Real}{Re}
\DeclareMathOperator{\Imag}{Im}
\DeclareMathOperator{\gradient}{grad}
\DeclareMathOperator{\divergence}{div}
\DeclareMathOperator{\rotor}{rot}
\DeclareMathOperator{\trace}{Tr}

\renewcommand{\headrulewidth}{0pt}

\makeatother

\usepackage{listings}
\renewcommand{\lstlistingname}{\inputencoding{latin7}Listing}

\begin{document}
\global\long\def\imath{\mathrm{i}}%
 
\global\long\def\ddiff{\mathrm{d}}%
 
\global\long\def\d{\mathrm{d}}%
 
\global\long\def\vect#1{\mathbf{#1}}%
 
\global\long\def\oper#1{\hat{#1}}%
 
\global\long\def\emath{\mathrm{e}}%
 

\global\long\def\ket#1{\left|#1\right\rangle }%
\global\long\def\bra#1{\left\langle #1\right|}%
\global\long\def\braket#1#2{\left\langle #1\middle|#2\right\rangle }%
 
\global\long\def\braketop#1#2#3{\left\langle #1\middle|#2\middle|#3\right\rangle }%
\global\long\def\si#1#2{\SI{#1}{#2}}%


\rhead{}

\lhead{}

\cfoot{\thepage}

\normalfont

\thispagestyle{empty}
\begin{center}
Vilniaus universitetas\\
Fizikos fakultetas\\
Dirbtinis Intelektas
\par\end{center}

\vspace{3.5cm}

\begin{center}
ZUIKIO KLAJONĖS\vspace{0.8cm}
\par\end{center}

\begin{center}
\vspace{0.8cm}
Teorinė fizika ir astrofizika
\par\end{center}

\begin{center}
\vspace{3.5cm}
\par\end{center}

\begin{flushright}
\begin{tabular}{cr}
Ruošė: & Paulius Juodsnukis\tabularnewline
 & Ieva Gugaitė\tabularnewline
\end{tabular}
\par\end{flushright}

\begin{center}
\vspace{4cm}
\par\end{center}

\begin{center}
VU FF 2021 01 27
\par\end{center}

\newpage{}

\normalfont\tableofcontents{}

\newpage{}

\section{Idiegimas}

Projektas naudoja dvi išorines Python bibleotekas: tkinter ir pygame.
Žiurėkite \ref{tab:lib table}. Pygame bibleotekoje naudojamas funkcionalumas
kuris leidžia sujungti šių dviejų bibleotekų veikimą viename lange.
Naujausiose pygame atnaujinimuose jis yra pašalintas, todėl naudojama
truputi senesnė 1.9.6 versija.

\begin{table}
\begin{centering}
\begin{tabular}{|c|c|l|}
\hline 
Bibleoteka & Versija & Komanda\tabularnewline
\hline 
tkinter & naujausia & \inputencoding{latin7}
\begin{lstlisting}
pip install tkinter
\end{lstlisting}
\inputencoding{utf8}\tabularnewline
\hline 
pygame & $\leq1.9.6$ & \inputencoding{latin7}
\begin{lstlisting}
pip install -Iv pygame==1.9.6
\end{lstlisting}
\inputencoding{utf8}\tabularnewline
\hline 
\end{tabular}
\par\end{centering}
\caption{\label{tab:lib table}Išorinės bibleotekos}
\end{table}


\section{Programos struktūra}

Projektas sudarytas iš eilės Python failų:
\begin{itemize}
\item adventure.py
\item configurations.py
\item data.py
\item program.py
\item qsolver.py
\item solvers.py
\item sprites.py
\item tests.py
\item utils.py
\item window.py
\end{itemize}
Ir dvieju direktorijų:
\begin{itemize}
\item pics
\item logs
\end{itemize}
Pirmojoje saugomi mūsų veikėjų: zuikio, vilkio ir morkos paveikslai
reikalingi atvaizduojant zuikio klajonę. Antrajame generuojami ir
saugomi log failai skirti išsaugoti ir tyrinėti zuikio išmoktas būsenas
ir jų perėjimus.

\subsection{Paleidimas}

\textcolor{red}{Kolkas} pagrindinis Python failas yra \textcolor{red}{tests.py}.
Jame parašyti testai leidžia bandyti ir testuoti įvairius programos
aspektus, įskaitant ir pati zuikio mokymais ir jo klajones.

Paleidus šį failą pasileis viena iš dvieju rutinų: nurodytasis testas
arba testų/profiliavimo pasirinkimas. Kad pasileistų pirmasis varijantas,
numatytasis testas turi buti nurodytas leidžiant testavimo subrutiną
kintamajame:\inputencoding{latin7}
\begin{lstlisting}
default_test
\end{lstlisting}
\inputencoding{utf8}
Jei jis nenurodytas(yra None) tada pasileis testų meniu.

\textcolor{red}{Šiuo laiko momentu dalis testų tikriausia neveikia
ir keli tikrai neveikia. Tačiau tikrai veikia paskutinis testas:}

\textcolor{red}{}\inputencoding{latin7}
\begin{lstlisting}
test_qsolver
\end{lstlisting}
\inputencoding{utf8}
\clearpage
\end{document}
